\documentclass{article}

\usepackage{amsmath, amssymb}
\usepackage{listings}
\usepackage{xcolor}
\usepackage{hyperref}

\definecolor{codegreen}{rgb}{0,0.6,0}
\definecolor{codegray}{rgb}{0.5,0.5,0.5}
\definecolor{codepurple}{rgb}{0.58,0,0.82}
\definecolor{backcolour}{rgb}{0.95,0.95,0.92}

\lstdefinestyle{mystyle}{
	commentstyle=\color{codegreen},
	keywordstyle=\color{magenta},
	stringstyle=\color{codepurple},
	basicstyle=\ttfamily\footnotesize,
	breakatwhitespace=false,
	captionpos=b,
	keepspaces=true,
	showspaces=false,
	showstringspaces=false,
	showtabs=false,
	tabsize=2
}

\lstset{style=mystyle}

\title{Lab assignment 4 report}
\author{Xiaoyue Chen}

\begin{document}

\maketitle

\section{}
\begin{align*}
	 & \quad \begin{bmatrix}
		1  & 5  & \vline & 7  \\
		-2 & -7 & \vline & -5
	\end{bmatrix}
	\begin{matrix}
		\\
		+2R_1
	\end{matrix}          \\
	 & \sim
	\begin{bmatrix}
		1 & 5 & \vline & 7 \\
		0 & 3 & \vline & 9
	\end{bmatrix}
\end{align*}
Hence $\boldsymbol{x}=(-8, 3)^\intercal$.

\section{}
\begin{align*}
	 & \quad \begin{bmatrix}
		0  & 2  & 1  & \vline & -8 \\
		1  & -2 & -3 & \vline & 0  \\
		-1 & 1  & 2  & \vline & 3
	\end{bmatrix}
	\begin{matrix}
		=R_3 \\
		\\
		=R1
	\end{matrix}          \\
	 & \sim \begin{bmatrix}
		-1 & 1  & 2  & \vline & 3  \\
		1  & -2 & -3 & \vline & 0  \\
		0  & 2  & 1  & \vline & -8
	\end{bmatrix}
	\begin{matrix}
		\\
		+R_1 \\
		\\
	\end{matrix}          \\
	 & \sim \begin{bmatrix}
		-1 & 1  & 2  & \vline & 3  \\
		0  & -1 & -1 & \vline & 3  \\
		0  & 2  & 1  & \vline & -8
	\end{bmatrix}
	\begin{matrix}
		\\
		\\
		+2R_2
	\end{matrix}          \\
	 & \sim \begin{bmatrix}
		-1 & 1  & 2  & \vline & 3  \\
		0  & -1 & -1 & \vline & 3  \\
		0  & 0  & -1 & \vline & -2
	\end{bmatrix}
\end{align*}
Hence $\boldsymbol{x} = (-4, -5, 2)^\intercal$

\section{}
\begin{align*}
	 & \quad \begin{bmatrix}
		1 & -2 & -6  & \vline & 12  \\
		2 & 4  & 12  & \vline & -17 \\
		1 & -4 & -12 & \vline & 22
	\end{bmatrix}
	\begin{matrix}
		\\
		-2R_1 \\
		-R_1
	\end{matrix}          \\
	 & \sim \begin{bmatrix}
		1 & -2 & -6 & \vline & 12  \\
		0 & 8  & 24 & \vline & -41 \\
		0 & -2 & -6 & \vline & 10
	\end{bmatrix}
	\begin{matrix}
		\\
		\\
		+\frac{R_2}{4}
	\end{matrix}          \\
	 & \sim \begin{bmatrix}
		1 & -2 & -6 & \vline & 12          \\
		0 & 8  & 24 & \vline & -41         \\
		0 & 0  & 0  & \vline & \frac{1}{4}
	\end{bmatrix}
\end{align*}
Hence the system is inconsistent.

\section{}
If you would like to run the file without awkwardly copying it from this pdf,
go to \url{https://github.com/xiaoyuechen/numerical-analysis/tree/master/lab4}
\lstinputlisting[language=Python]{gaussian_elimination.py}
\end{document}
